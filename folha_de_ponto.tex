%! Autor = Jander Moreira
%%%%%%%%%%%%%%%%%%%%%%%%%%%%%%%%%%%%%%%%%%%%%%%%%%%%%%%%%%%%%%%%%%%%%%%%%%%%
%% Instruções de uso (versão fácil):
%%
%%  1) Visite overleaf.com (necessário registro e login)
%%  2) Clique em 'Novo projeto' ('New Project') e crie um projeto em branco
%%      (blank)
%%  3) Dê ao projeto o nome que quiser
%%  4) Copie este documento inteiro e substitua o texto padrão que está lá
%%  5) Preencha os dados (logo abaixo neste documento), tomando cuidado de
%%      não apagar as chaves ou as vírgulas que circundam os dados
%%  6) Clique em 'Recompilar' ('Recompile')
%%  7) Verifique os dados e corrija quando necessário; atualize o mês e
%%      o ano para a data que deseja
%%  7) Clique no ícone para baixar o PDF criado (logo ao lado de
%%      'Recompilar')
%%  8) Imprima o PDF baixado
%%
%%  O mesmo documento pode ser editado no Overleaf nas próximas vezes,
%%      apenas alterando-se mês e ano
%%
%%%%%%%%%%%%%%%%%%%%%%%%%%%%%%%%%%%%%%%%%%%%%%%%%%%%%%%%%%%%%%%%%%%%%%%%%%%%
\documentclass[12pt, a4paper]{article}
\usepackage[utf8]{inputenc}
\usepackage[T1]{fontenc}
\usepackage[brazilian]{babel}
\usepackage[brazilian]{translator}
\usepackage{tikz}
\usetikzlibrary{calendar, shapes.multipart, positioning}
\pgfkeys{
    /folha/.cd,
    empregador/.store in = \NomeEmpregador,
    documentos empregador/.store in = \DocumentosEmpregador,
    empregado/.store in = \NomeEmpregado,
    documentos empregado/.store in = \DocumentosEmpregado,
    mes/.store in = \MesAtual,
    ano/.store in = \AnoAtual,
}


%%%%%%%%%%%%%%%%%%%%%%%%%%%%%%%%%%%%%%%%%%%%%%%%%%%%%%%%%%%%%%%%%%%%%%%%%%%%
%%%        EDITE APENAS AQUI   :-)
%
%  * Mantenha as chaves em torno dos nomes e documentos
%
\pgfkeys{
    /folha/.cd,
    empregador = {Fulano de Tal},
    documentos empregador = {CPF: 123.321.999-00},
    empregado = {Ciclana de Outrotal},
    documentos empregado = {NIT: 123.123.321-99, CTPS: 23123-9},
    mes = 7,
    ano = 2023,
}
%%%%%%%%%%%%%%%%%%%%%%%%%%%%%%%%%%%%%%%%%%%%%%%%%%%%%%%%%%%%%%%%%%%%%%%%%%%%


\usepackage{geometry}
\geometry{left = 1.5cm, right = 1.5cm, top = 1.25cm, bottom = 1cm}

\tikzset{
    dia/.style = {
        draw,
        thick,
        anchor = west,
        align = left,
        text width = 3.4cm,
        minimum height = 0.65cm,
        font = \small\sffamily,
        inner sep = 0pt,
    },
    celula/.style = {
        draw,
        thick,
        text width = 2.4cm,
        minimum height = 0.65cm,
        font = \small\sffamily\bfseries,
        inner sep = 0pt,
        align = center,
    },
    folha/.style = {
        day list downward,
        day code = {
            \node[dia] (node) {~Dia \tikzdaytext\ -- \DiaDaSemana\strut};
            \foreach \i in {1, ..., 6}{
                \node[celula, right = of node] (node) {\Texto\strut};
            }
        },
        dates = \AnoAtual-\MesAtual-01 to \AnoAtual-\MesAtual-last,
        day yshift = 0.65cm,
    },
    node distance = 0.0pt,
}

\tikzset{
    /folha/dia da semana/.store in = \DiaDaSemana,
    /folha/texto/.store in = \Texto,
}

\usepackage{tcolorbox}
\tcbset{
    colback = white,
}
\usepackage[portuges]{datetime2}


\begin{document}
\thispagestyle{empty}


\begin{tcolorbox}
    Empregador: \textbf{\NomeEmpregador}\par
    Documentos: \textbf{\DocumentosEmpregador}
    \tcblower
    Empregada: \textbf{\NomeEmpregado}\par
    Documentos: \textbf{\DocumentosEmpregado}
\end{tcolorbox}

\vspace{-0.55cm}
\begin{tcolorbox}
    \centering\large\bfseries
    Folha de Ponto --
    \MakeUppercase{\DTMportugesmonthname{\MesAtual}/\AnoAtual}
\end{tcolorbox}

\medskip
\centering
\begin{tikzpicture}
    \node[dia, draw = none,  minimum height = 1.4cm] (node) {\strut};
    \foreach \txt in {Entrada\\(manhã), Saída\\(manhã),
    Entrada\\(tarde), Saída\\(tarde), Rubrica\\Empregador, Rubrica\\Empregado}{
        \node[celula, right = of node, minimum height = 1.4cm] (node) {\txt\strut};
    }
\end{tikzpicture}\\
\begin{tikzpicture}
    \calendar[folha]
    if (Sunday) [red!80!black, /folha/dia da semana = Domingo, /folha/texto = ---]
    if (Monday) [/folha/dia da semana = Segunda, /folha/texto = ]
    if (Tuesday) [/folha/dia da semana = Terça, /folha/texto = ]
    if (Wednesday) [/folha/dia da semana = Quarta, /folha/texto = ]
    if (Thursday) [/folha/dia da semana = Quinta, /folha/texto = ]
    if (Friday) [/folha/dia da semana = Sexta, /folha/texto = ]
    if (Saturday) [red!80!black, /folha/dia da semana = Sábado, /folha/texto = ---]
    ;
\end{tikzpicture}

\vfill\noindent
\hrulefill

\end{document}
